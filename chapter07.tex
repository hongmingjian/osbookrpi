\documentclass[main.tex]{subfiles}
\ifxetex\else\onlyinsubfile{\usepackage{CJKutf8}}\fi
\begin{document}
\ifxetex\else\begin{CJK*}{UTF8}{song}\fi

\chapter{用户程序与系统调用}
不知不觉,我们已经完成了许多工作,包括中断和异常处理、内存管理、多线程和文件系统。到目前为止,所有的工作都是在内核中进行的。本章将进入用户模式运行用户程序,而且在用户程序中通过系统调用请求内核的一些功能,比如创建新的用户线程、操作文件等。

\section{启动用户程序}
在第5章中说过, run\_\-as\_\-task0之后的代码,都是以 task0的身份运行。所以在上一章末尾, task0初始化了 SD卡、挂载 FAT文件系统等。接下来, task0
会从 FAT的根目录中加载程序文件 a.out,然后创建第一个用户线程执行该程序的函数 main。

\par
加载 a.out是由函数 load\_\-aout(定义于文件 kernel/elf.c)完成的,一切正常时它会返回程序的入口地址。接着, task0用 page\_\-alloc\_\-in\_\-addr给用户线程准备栈,它位于 USER\_\-MAX\_\-ADDR(=0xbfc0\-0000),大小 1MiB,也是朝下增长。这里只给栈分配了地址空间,并未分配物理内存。线程运行后,访问栈上的数据时,通过缺页异常自动完成物理内存分配。

\par
然后调用 sys\_\-create\_\-task创建线程,而且给线程传递 0x2020\-0416作为参数。创建成功后,该用户线程会从 a.out的入口地址开始执行。而 task0会被困在 566~ 567行的循环中无法自拔。

\begin{code}
\captionof{listing}{chapter07/kernel/machdep.c}
\label{code:7-1}
\inputminted[firstline=538,lastline=570,linenos,numbersep=5pt,frame=lines,framesep=2mm]{c}{src/chapter07/kernel/machdep.c}
\end{code}

\section{用户程序}
用户程序文件 a.out是以 \htmladdnormallink{ELF}{https://elinux.org/Executable_and_Linkable_Format_(ELF)}格式存储的可执行文件。它是为我们的操作系统开发的应用程序,只能保证在这个操作系统上能运行,在其他操作系统中一般是不可执行的。

\par
首先,新建一个目录 userapp,用于保存用户程序的源文件。因为内核与用户程序基本上是独立的,新建一个目录便于组织文件。它们共享的头文件放在 include目录中,共享的C文件放在目录 lib。 kernel和 userapp目录内部,有各自的 lib或 include目录,如图\ref{figure:7-1}所示。

\begin{figure}[htp]
\centering
\includegraphics[scale=0.5]{figures/7-1.png}
\caption{项目目录结构}
\label{figure:7-1}
\end{figure}

\subsection{userapp/lib/crt0.S}
文件 crt0.S包含了在函数 main之前运行的启动代码,这正是名字中“0”的意思。 crt是 C Run-Time的缩写,意味着它是 C运行时刻库的一部分。 crt0.S一般在运行用户定义的函数 main之前,做一些初始化的工作,然后调用 main。

\begin{code}
\captionof{listing}{chapter07/userapp/lib/crt0.S}
\label{code:7-2}
\inputminted[linenos,numbersep=5pt,frame=lines,framesep=2mm]{gas}{src/chapter07/userapp/lib/crt0.S}
\end{code}

线程启动时,会从 crt0.S定义的 \_start开始运行。因为这里没有什么初始化工作,所以直接调用 main。传给 \_start的参数,也会通过寄存器 r0传给 main。当 main返回后,调用 task\_\-exit结束线程,而且 main的返回值(在 r0中)也作为 task\_\-exit的参数。

\subsection{userapp/main.c}
下面来看 main函数。它首先打开串口设备“\$:/uart0”。第一次是以只读模式打开,后面两次以只写模式打开,分别对应于标准输入、输出和出错。此时系统尚未打开文件,所以这三个 open返回的文件描述符一定是 0、1、2。

\begin{code}
\captionof{listing}{chapter07/userapp/main.c}
\label{code:7-3}
\inputminted[firstline=52,lastline=83,linenos,numbersep=5pt,frame=lines,framesep=2mm]{c}{src/chapter07/userapp/main.c}
\end{code}

打开这三个标准文件后,就可以用 printf往串口打印信息了。该函数定义于文件 user\-app\-/lib\-/stdio.c中,它调用 vsn\-printf格式化字符串后,直接用 write把字符串写到标准输出即可。

\par
接下来, main打印线程号和参数 pv。然后,创建了三个线程 H、A、B。为了简单起见,用全局数组作为线程的栈,因为栈是朝下增长,要给 task\_\-create传递数组末尾的地址。第一个线程H表示 heart\-beat(心跳),它会让树莓派上名字为 ACT的 LED灯隔一秒钟闪一次,来作为系统的心跳指示。后面两个线程 A、 B与第5章相同,各打印 1000个 A和 B后退出。 main线程等待 A、 B退出后,进入一个无限循环:从标准输入读入一个字符,把它写到标准输出。

\par
心跳线程H的代码如\ref{code:7-4}所示。它打开设备文件“\$:/led0”,并用 ioctl去开关 LED。

\begin{code}
\captionof{listing}{chapter07/userapp/main.c}
\label{code:7-4}
\inputminted[firstline=35,lastline=50,linenos,numbersep=5pt,frame=lines,framesep=2mm]{c}{src/chapter07/userapp/main.c}
\end{code}

前面代码中的 open、 read、 write、 ioctl、 task\_\-create、 task\_\-getid、 task\_\-wait、 task\_\-exit、 sleep等都是系统调用(system call, syscall),将在\ref{7-syscall}节中介绍。

\subsection{userapp/Makefile}
用户程序的 Makefile并没有太多特别之处,这里主要看一下它的链接选项。

\begin{code}
\captionof{listing}{userapp/Makefile}
\label{code:7-5}
\inputminted[firstline=31,lastline=42,linenos,numbersep=5pt,frame=lines,framesep=2mm]{make}{src/chapter07/userapp/Makefile}
\end{code}

编译选项与内核并没有什么不同,仍然必须是 -ffree\-standing;链接器选项中,也必须是 -nostdlib。而选项“-Wl,-e,\_start,-Ttext,0x00400000”告诉链接器该可执行文件的入口符号是 \_start、代码段(即文本段)的起始地址是 0x0040\-0000(4MiB)。

\par
函数 load\_\-aout加载程序文件时,会把代码和数据加载到指定的内存地址,然后返回 \_start符号的地址作为线程函数的地址。

\section{系统调用}\label{7-syscall}
系统调用是操作系统对外提供服务的唯一入口。用户程序在用户模式运行,而内核运行于 supervisor模式。显然,用户程序不能直接调用内核的函数,必须通过模式切换后,在严密监控下进入内核。下面以 open为例,来跟踪系统调用的过程。

\subsection{用户接口}
当用户调用 open时,先调用了系统调用的用户接口,即 userapp/lib/syscall-wrapper.S。就如名字所表示的,它是系统调用的包裹,把系统调用包裹成C语言可以调用的方式。

\begin{code}
\captionof{listing}{chapter07/userapp/lib/syscall-wrapper.S}
\label{code:7-6}
\inputminted[linenos,numbersep=5pt,frame=lines,framesep=2mm]{gas}{src/chapter07/userapp/lib/syscall-wrapper.S}
\end{code}

它通过宏定义来生成用户接口,除了系统调用的名字和号码不同,其他完全相同。因此,用户接口把方便人类使用的名字,转换成对应的号码。也即是说,用户调用了 open,该宏定义把它转换为20号(=SYS\-CALL\_\-open的)系统调用。

\par
具体来说,用户接口把 r4备份在栈上,然后把系统调用号码放到 r4中,再用指令 swi进入内核。系统调用返回后,接着运行 swi之后的指令。为什么用 r4存放号码呢?根据 AAPCS (Procedure Call Standard for the ARM Architecture),函数的前四个参数分别保存在 r0-3中。如果还有更多参数,则通过栈传递。既然 r0-3已经预留给了系统调用的前四个参数,那就用 r4吧。

\par
系统调用的号码定义在文件 include/syscall-nr.h中,它是内核与用户程序共用的头文件,表示内核与用户程序约定了系统调用的号码。

\begin{code}
\captionof{listing}{chapter07/include/syscall-nr.h}
\label{code:7-7}
\inputminted[firstline=23,lastline=35,linenos,numbersep=5pt,frame=lines,framesep=2mm]{c}{src/chapter07/include/syscall-nr.h}
\end{code}

\subsection{进入内核}
下面进入内核。指令 swi把 CPU从用户模式切换到内核模式,然后跳转到异常向量表的第3项开始执行。我们在第4章已经打开了 Hi\-Vecs,所以会跳到 0xffff\-0008开始执行。注意,此时内核模式的 sp已经等于 (KER\-NBASE\-+0x1000\--36)。

\par
先从异常向量表开始。在 vector\_swi处填入标号 swi。

\begin{code}
\captionof{listing}{chapter07/kernel/entry.S}
\label{code:7-8}
\inputminted[firstline=131,lastline=156,linenos,numbersep=5pt,frame=lines,framesep=2mm]{gas}{src/chapter07/kernel/entry.S}
\end{code}

标号 swi的定义如\ref{code:7-9}所示,其中 246~ 253行把用户模式的寄存器以 struct context结构保存在全局内核栈上,然后用 PRO\-LOGUE把它复制到当前线程的内核栈,最后调用 C语言函数 syscall进行分发。指令 swi会关闭 CPU的 IRQ中断,因此在进入 syscall前把 IRQ打开,返回后关闭中断。

\begin{code}
\captionof{listing}{chapter07/kernel/entry.S}
\label{code:7-9}
\inputminted[firstline=245,lastline=262,linenos,numbersep=5pt,frame=lines,framesep=2mm]{gas}{src/chapter07/kernel/entry.S}
\end{code}

\subsection{系统调用的分发}
函数 syscall根据系统调用的号码,调用相应的功能函数,完成用户的请求。

\par
因为用户模式的寄存器已经被保存在 ctx中,因此要通过 ctx访问它们,即系统调用号码保存在 ctx-\textgreater r4中,前4个参数保存在 ctx-\textgreater r0-3中。如果有多于4个参数,要用 ctx-\textgreater usr\_sp去用户栈上取。注意,ctx-\textgreater usr\_sp处保存了备份的 r4(见代码\ref{code:7-7}),所以第5个参数的地址是 ctx-\textgreater usr\_sp+4,以此类推。

\begin{code}
\captionof{listing}{chapter07/kernel/machdep.c}
\label{code:7-10}
\inputminted[firstline=573,lastline=599,linenos,numbersep=5pt,frame=lines,framesep=2mm]{c}{src/chapter07/kernel/machdep.c}
\end{code}

\section{串口和LED}
前面的用户程序中打开了设备文件“\$:/uart0”和“\$:/led0”。下面看看它们在内核中的实现。

\subsection{uart0}
串口设备 uart0的实现很简单。因为内核已经配置好了串口,这里只要按照第6章的设备接口进行封装即可。

\begin{code}
\captionof{listing}{chapter07/kernel/uart.c}
\label{code:7-11}
\inputminted[firstline=15,lastline=72,linenos,numbersep=5pt,frame=lines,framesep=2mm]{c}{src/chapter07/kernel/uart.c}
\end{code}

\subsection{led0}
在树莓派电路板上可以找到名字标为 ACT的 LED灯,它连接在某个 GPIO口上。因此,把 GPIO口配置为输出,然后控制电平即可开关。不同版本的树莓派, ACT连接到不同的 GPIO口,而且高低电平定义也不同。所以,结构体 led\_dev中包含了 ACT挂在哪个 GPIO口上,以及高低电平对应的开关信息。

\begin{code}
\captionof{listing}{chapter07/kernel/led.c}
\label{code:7-12}
\inputminted[firstline=20,lastline=80,linenos,numbersep=5pt,frame=lines,framesep=2mm]{c}{src/chapter07/kernel/led.c}
\end{code}

\section{运行结果}
本章完成后,运行结果如图\ref{figure:7-2}所示,其中最后一行是键盘输入的字符。除此之外,可以在树莓派上看到 ACT灯在规则闪烁。

\begin{figure}[htp]
\centering
\includegraphics[scale=0.5]{figures/7-2.png}
\caption{运行结果}
\label{figure:7-2}
\end{figure}

\clearpage
\ifxetex\else\end{CJK*}\fi
\end{document}
